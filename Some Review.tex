\documentclass[10pt,notitlepage]{article}
\usepackage[left=2cm, right=2cm, top=2cm]{geometry}
\usepackage[utf8]{inputenc}
\usepackage{amssymb}
\usepackage{amsthm} 
\usepackage{amsmath}
\usepackage{tikz-cd}

\title{\vspace{-2cm}Some Algebraic Topology Review\vspace{-1cm}}
\date{}
\author{}

\begin{document}
\maketitle	

\section{Git hello}

	\begin{enumerate}
		\item Change
		\item BLAH For fixed $i,n \geq 1$ construct a space, $X$, with $\widetilde{H}_{i}(X) = \mathbb{Z}/n\mathbb{Z}$ and $\widetilde{H}_{j}(X) = 0$ for $j \neq i$.
		\item \begin{enumerate}
			\item  Compute the co/homology groups with $\mathbb{Z}$, $\mathbb{R}$ and $\mathbb{Z}/2\mathbb{Z}$ coefficients of $X = \mathbb{R}^{3} - A$ where $A = B \cup C$, $B$ is the $z$-axis, and $C$ is the unit circle in the $xy$-plane via a Mayer-Vietoris long exact sequence.
			\item Describe how $X$ is homotopy equivalent to a torus.
		\end{enumerate}
	\item Do the functors $h_{n}(X) = H_{n+2}(X) \oplus H_{n-3}(X)$ describe a homology theory? Do the functors $h^{n} = \text{Hom}(H_{n}(X),\mathbb{Z})$ describe a cohomology theory?
	\item What is the mistake in the following calculation of $H_{1}(S^{1})$?\\
	We can consider the circle $S^{1}$ to be the union of a closed upper semicircle $X$ and open lower semicircle $Y$. Both of these are contractible, so $\widetilde{H}_{n}(X) = \widetilde{H}_{n}(Y) = 0$. Further $X \cap Y = \emptyset$, so $\widetilde{H}_{n}(X \cap Y) = 0$. Then from the Mayer-Vietoris long exact sequence we have
	$$ \cdots \rightarrow H_{1}(X \cap Y) \rightarrow H_{1}(X) \oplus H_{1}(Y) \rightarrow H_{1}(S^{1}) \rightarrow H_{0}(X \cap Y) \rightarrow \cdots. $$
	which yields $0 \rightarrow H_{1}(S^{1}) \rightarrow 0$. Then by exactness $H_{1}(S^{1}) = 0$.
	\item Remember the fun workshop exercise (1.2.14 in Hatcher) where we calculated the fundamental group of $I^{3}$ with opposite faces identified under a clockwise one-quarter twist? Calculate the co/homology of this quotient space.
	\item Show that the commutative diagram
	\begin{center}
		\begin{tikzcd}[row sep = tiny, ampersand replacement = \&]
			\cdots \ar[r] \& C_{n+1} \ar[dd] \ar[dr] \& \& B_{n} \ar[r, "\epsilon_{n}"] \ar[dd, swap, "\delta_{n}"] \& C_{n} \ar[dd, "\pi_{n}"] \ar[dr, "\mu_{n}"] \& \& B_{n-1} \ar[dd] \ar[r] \& \cdots \\
			\& \& A_{n} \ar[ur, "\alpha_{n}"] \ar[dr, swap, "\beta_{n}"] \& \& \& A_{n-1} \ar[ur] \ar[dr] \& \& \\
			\cdots \ar[r] \& E_{n+1} \ar[ur] \& \& D_{n} \ar[r, swap, "\gamma_{n}"] \& E_{n} \ar[ur, swap, "\eta_{n}"] \& \& D_{n-1} \ar[r] \& \cdots \\
		\end{tikzcd}
	\end{center}
	with the two sequences across the top and bottom exact, gives rise to an exact sequence $$\cdots \rightarrow E_{n+1} \rightarrow B_{n} \rightarrow C_{n} \oplus D_{n} \rightarrow E_{n} \rightarrow B_{n-1} \rightarrow \cdots$$
	where the maps are obtained from those in the previous diagram in the obvious way, except that $B_{n} \rightarrow C_{n} \oplus D_{n}$ has a minus sign in one coordinate.
	\item For the mapping torus $T_{f}$ of a map $f:X \rightarrow X$, Example 2.48 in Hatcher constructs a long exact sequence $$\cdots \rightarrow H_{n}(X) \xrightarrow{1-f_{\ast}} H_{n}(X) \xrightarrow{\iota_{n}} H_{n}(T_{f}) \xrightarrow{\delta_{n}} H_{n-1}(X) \rightarrow \cdots.$$ Use this to compute the homology of the mapping tori following maps:
		\begin{enumerate}
			\item [(a)] A reflection $S^{2} \rightarrow S^{2}$.
			
			\item [(b)] A map $S^{2} \rightarrow S^{2}$ of degree 2.
			
			\item [(c)] The map $S^{1} \times S^{1} \rightarrow S^{1} \times S^{1}$ that is the identity on one factor and a reflection on the other. 
			
			\item [(d)] The map $S^{1} \times S^{1} \rightarrow S^{1} \times S^{1}$ that is a reflection on each factor.
			
			\item [(e)] The map $S^{1} \times S^{1} \rightarrow S^{1} \times S^{1}$ that interchanges the two factors and then reflects one of the factors.
		\end{enumerate}
	\item Construct the universal covering spaces of the figure 8, the torus, the Klein bottle, and the
	projective plane, and the wedge sum of two projective planes. Prove that the first three are
	contractible.
	\end{enumerate}
\end{document}